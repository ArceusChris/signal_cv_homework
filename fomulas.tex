\documentclass[10pt,a4paper]{article}
\usepackage[UTF8]{ctex}
\usepackage{amsmath,amssymb}
\usepackage{geometry}
\usepackage{multicol}
\geometry{left=0.5cm,right=0.5cm,top=0.5cm,bottom=0.5cm}
\setlength{\parindent}{0pt}
\setlength{\parskip}{0pt}
\setlength{\columnseprule}{0.2pt}
\begin{document}
\begin{multicols}{3}
\textbf{一、傅里叶级数}
\scriptsize

\textbf{连续周期信号:}$f(t) = f(t+T)$,$\omega_0 = \frac{2\pi}{T}$

\textbf{三角形式:}$f(t) = a_0 + \sum_{n=1}^{\infty}[a_n\cos(n\omega_0 t) + b_n\sin(n\omega_0 t)]$

\textbf{指数形式:}$f(t) = \sum_{n=-\infty}^{\infty}c_n e^{jn\omega_0 t}$,其中
$$c_n = \frac{1}{T}\int_T f(t)e^{-jn\omega_0 t}dt$$

\textbf{关系:}$c_0 = a_0$,$c_n = \frac{a_n-jb_n}{2}$,$c_{-n} = c_n^*$($f(t)$实)
\begin{itemize}
\itemsep=0pt
\item 线性:$af(t)+bg(t) \leftrightarrow ac_n+bd_n$
\item 时移:$f(t-t_0) \leftrightarrow c_n e^{-jn\omega_0 t_0}$
\item 共轭对称:$f(t)$实 $\Rightarrow c_{-n}=c_n^*$
\item 帕塞瓦尔:$\frac{1}{T}\int_T|f(t)|^2dt = \sum_{n=-\infty}^{\infty}|c_n|^2$
\end{itemize}


\noindent\textbf{二、傅里叶变换}
\scriptsize
\begin{align*}
F(\omega) &= \mathcal{F}[f(t)] = \int_{-\infty}^{\infty} f(t)e^{-j\omega t}dt \\
f(t) &= \mathcal{F}^{-1}[F(\omega)] = \frac{1}{2\pi}\int_{-\infty}^{\infty} F(\omega)e^{j\omega t}d\omega
\end{align*}
\scriptsize
\begin{enumerate}
\itemsep=0pt
\item \textbf{线性性:} $\mathcal{F}[af(t)+bg(t)] = aF(\omega)+bG(\omega)$
\item \textbf{时移性:} $\mathcal{F}[f(t-t_0)] = F(\omega)e^{-j\omega t_0}$
\item \textbf{频移性:} $\mathcal{F}[f(t)e^{j\omega_0 t}] = F(\omega-\omega_0)$
\item \textbf{尺度变换:} $\mathcal{F}[f(at)] = \frac{1}{|a|}F(\frac{\omega}{a})$
\item \textbf{对称性(对偶性):} $\mathcal{F}[F(t)] = 2\pi f(-\omega)$
\item \textbf{时域微分:} $\mathcal{F}[\frac{d^n f(t)}{dt^n}] = (j\omega)^n F(\omega)$
\textbf{条件:}$f(-\infty)+f(\infty)=0$
\item \textbf{频域微分:} $\mathcal{F}[t^n f(t)] = j^n \frac{d^n F(\omega)}{d\omega^n}$
\item \textbf{积分性:} $\mathcal{F}[\int_{-\infty}^{t}f(\tau)d\tau] = \frac{F(\omega)}{j\omega} + \pi F(0)\delta(\omega)$
\item \textbf{时域卷积:} $\mathcal{F}[f(t)*g(t)] = F(\omega)G(\omega)$
\item \textbf{频域卷积:} $\mathcal{F}[f(t)g(t)] = \frac{1}{2\pi}F(\omega)*G(\omega)$
\item \textbf{能量与功率:}
   \begin{itemize}
   \itemsep=0pt
   \item 能量信号:$E = \int_{-\infty}^{\infty}|f(t)|^2dt = \frac{1}{2\pi}\int_{-\infty}^{\infty}|F(\omega)|^2d\omega$
   \item 功率信号(周期):$P = \frac{1}{T}\int_T|f(t)|^2dt = \sum_{n=-\infty}^{\infty}|a_n|^2$ ($a_n$为傅里叶系数)
   \item 平均功率:$P = \lim_{T\to\infty}\frac{1}{T}\int_{-T/2}^{T/2}|f(t)|^2dt$
   \end{itemize}
\item \textbf{共轭对称性:}若$f(t)$实,则$F(-\omega) = F^*(\omega)$
\item \textbf{奇偶虚实性:}
   \begin{itemize}
   \itemsep=0pt
   \item $f(t)$实偶 $\Rightarrow$ $F(\omega)$实偶
   \item $f(t)$实奇 $\Rightarrow$ $F(\omega)$纯虚奇
   \item $f(t)$虚偶 $\Rightarrow$ $F(\omega)$虚偶
   \item $f(t)$虚奇 $\Rightarrow$ $F(\omega)$实奇
   \end{itemize}
\item \textbf{奇偶分解:}
   \begin{itemize}
   \itemsep=0pt
   \item 偶部:$f_e(t) = \frac{f(t)+f(-t)}{2}$ $\Leftrightarrow$ $F_e(\omega) = \text{Re}\{F(\omega)\}$
   \item 奇部:$f_o(t) = \frac{f(t)-f(-t)}{2}$ $\Leftrightarrow$ $F_o(\omega) = j\text{Im}\{F(\omega)\}$
   \item 关系:$f(t) = f_e(t) + f_o(t)$,$F(\omega) = F_e(\omega) + F_o(\omega)$
   \end{itemize}
\end{enumerate}

\noindent\textbf{三、常见傅里叶变换对}
\scriptsize

\textbf{常用函数定义:}
\begin{align*}
\text{rect}(t) & \text{ 宽为1} \\
\text{tri}(t) & \text{ 宽为2} \\
\text{sinc}(t) & = \frac{\sin(\pi t)}{\pi t} \\
\text{sa}(t) & = \frac{\sin(t)}{t}
\end{align*}

\begin{align*}
&\delta(t) \longleftrightarrow 1 \\
&1 \longleftrightarrow 2\pi\delta(\omega) \\
&e^{j\omega_0 t} \longleftrightarrow 2\pi\delta(\omega-\omega_0) \\
&\cos(\omega_0 t) \longleftrightarrow \pi[\delta(\omega-\omega_0)+\delta(\omega+\omega_0)] \\
&\sin(\omega_0 t) \longleftrightarrow j\pi[\delta(\omega+\omega_0)-\delta(\omega-\omega_0)] \\
&e^{-at}u(t) \longleftrightarrow \frac{1}{a+j\omega}, \quad (a>0) \\
&u(t) \longleftrightarrow \pi\delta(\omega) + \frac{1}{j\omega} \\
\end{align*}
\begin{align*}
&\text{rect}(\frac{t}{\tau}) \longleftrightarrow \frac{\tau\sin(\omega\tau/2)}{\omega\tau/2} \\
&\text{sinc}(t) = \frac{\sin(\pi t)}{\pi t} \longleftrightarrow \text{rect}(\frac{\omega}{2\pi}) \\
&\text{tri}(\frac{t}{\tau}) \longleftrightarrow \frac{\tau\sin^2(\omega\tau/2)}{(\omega\tau/2)^2} \\
&\sum_{n=-\infty}^{\infty}\delta(t-nT) \longleftrightarrow \frac{2\pi}{T}\sum_{k=-\infty}^{\infty}\delta(\omega-k\omega_0)
\end{align*}
\textbf{周期信号的傅里叶变换:}

\textbf{方法一(利用傅里叶级数):}

若$f(t) = \sum_{n=-\infty}^{\infty}c_n e^{jn\omega_0 t}$,则
$$F(\omega) = 2\pi\sum_{n=-\infty}^{\infty}c_n\delta(\omega-n\omega_0)$$

\textbf{方法二(单周期信号法):}

$$f(t) = \sum_{k=-\infty}^{\infty}f_0(t-kT)$$

$$F(\omega) = \frac{2\pi}{T}F_0(\omega)\sum_{n=-\infty}^{\infty}\delta(\omega-n\omega_0)$$

其中$c_n = \frac{1}{T}F_0(n\omega_0)$

\noindent\textbf{四、离散时间傅里叶变换(DTFT)}
\scriptsize
\begin{align*}
X(e^{j\omega}) &= \sum_{n=-\infty}^{\infty}x[n]e^{-j\omega n} \\
x[n] &= \frac{1}{2\pi}\int_{-\pi}^{\pi}X(e^{j\omega})e^{j\omega n}d\omega
\end{align*}
\begin{itemize}
\itemsep=0pt
\item 线性:$\mathcal{F}_d[ax[n]+by[n]] = aX(e^{j\omega})+bY(e^{j\omega})$
\item 时移:$\mathcal{F}_d[x[n-n_0]] = e^{-j\omega n_0}X(e^{j\omega})$
\item 频移:$\mathcal{F}_d[e^{j\omega_0 n}x[n]] = X(e^{j(\omega-\omega_0)})$
\item 周期性:$X(e^{j(\omega+2\pi)}) = X(e^{j\omega})$
\item 共轭对称:若$x[n]$实,则$X(e^{-j\omega}) = X^*(e^{j\omega})$
\item 频域微分:$\mathcal{F}_d[nx[n]] = j\frac{dX(e^{j\omega})}{d\omega}$
\item 时域扩展:$x_k[n] = \begin{cases} x[n/k], & n=0,\pm k,\pm 2k,\ldots \\ 0, & \text{其他} \end{cases}$,则$X_k(e^{j\omega}) = X(e^{jk\omega})$
\item 卷积:$\mathcal{F}_d[x[n]*y[n]] = X(e^{j\omega})Y(e^{j\omega})$
\item 调制:$\mathcal{F}_d[x[n]y[n]] = \frac{1}{2\pi}X(e^{j\omega})*Y(e^{j\omega})$
\item 奇偶虚实性:
   \begin{itemize}
   \itemsep=0pt
   \item $x[n]$实偶 $\Rightarrow$ $X(e^{j\omega})$实偶
   \item $x[n]$实奇 $\Rightarrow$ $X(e^{j\omega})$纯虚奇
   \item $x[n]$虚偶 $\Rightarrow$ $X(e^{j\omega})$虚偶
   \item $x[n]$虚奇 $\Rightarrow$ $X(e^{j\omega})$实奇
   \end{itemize}
\item 低频和高频:低频指$\omega\approx 2k\pi$,高频指$\omega\approx (2k+1)\pi$
\end{itemize}

\begin{align*}
&\delta[n] \longleftrightarrow 1 \\
&\delta[n-n_0] \longleftrightarrow e^{-j\omega n_0} \\
&a^n u[n] \longleftrightarrow \frac{1}{1-ae^{-j\omega}}, \quad |a|<1 \\
&u[n] \longleftrightarrow \frac{1}{1-e^{-j\omega}} + \pi\sum_{k=-\infty}^{\infty}\delta(\omega-2\pi k) \\
&\text{rect}_N[n] \longleftrightarrow \frac{\sin(\omega N/2)}{\sin(\omega/2)}e^{-j\omega(N-1)/2} \\
H_{\text{LP}}(e^{j\omega}) &= \begin{cases}
1, & |\omega| \leq \omega_c \\
0, & \omega_c < |\omega| \leq \pi
\end{cases} \\
h_{\text{LP}}[n] &= \frac{\omega_c}{\pi} \cdot \frac{\sin(\omega_c n)}{\omega_c n} = \frac{\sin(\omega_c n)}{\pi n}
\end{align*}
其中$\omega_c$为截止频率,$h_{\text{LP}}[n]$为理想低通滤波器的冲激响应。

\noindent\textbf{五、拉普拉斯变换(双边)}
\scriptsize
\textbf{定义:}$F(s) = \mathcal{L}[f(t)] = \int_{-\infty}^{\infty}f(t)e^{-st}dt, \quad s=\sigma+j\omega$
\begin{itemize}
\itemsep=0pt
\item 线性:$\mathcal{L}[af(t)+bg(t)] = aF(s)+bG(s)$,ROC至少为$R_1\cap R_2$
\item 时移:$\mathcal{L}[f(t-t_0)] = e^{-st_0}F(s)$,ROC不变
\item 频移:$\mathcal{L}[e^{-at}f(t)] = F(s+a)$,ROC:$\text{Re}(s+a)\in R$
\item 尺度:$\mathcal{L}[f(at)] = \frac{1}{|a|}F(\frac{s}{a})$,ROC:$\frac{s}{a}\in R$
\item 时域微分:$\mathcal{L}[f'(t)] = sF(s)$,ROC至少为$R$
\item s域微分:$\mathcal{L}[tf(t)] = -\frac{dF(s)}{ds}$,ROC:$R$
\item 积分:$\mathcal{L}[\int_{-\infty}^{t} f(\tau)d\tau] = \frac{F(s)}{s}$,ROC至少为$R\cap\{\text{Re}(s)>0\}$
\item 卷积:$\mathcal{L}[f(t)*g(t)] = F(s)G(s)$,ROC至少为$R_1\cap R_2$
\item 初值定理:$f(0^+) = \lim_{s\to\infty}sF(s)$ (因果信号)
\item 终值定理:$\lim_{t\to\infty}f(t) = \lim_{s\to0}sF(s)$ (极点在左半平面或原点)
\end{itemize}
\begin{align*}
&\delta(t) \longleftrightarrow 1, \quad \text{全}s\text{平面} \\
&u(t) \longleftrightarrow \frac{1}{s}, \quad \text{Re}(s)>0 \\
&-u(-t) \longleftrightarrow \frac{1}{s}, \quad \text{Re}(s)<0 \\
&e^{-at}u(t) \longleftrightarrow \frac{1}{s+a}, \quad \text{Re}(s)>-a \quad \text{(右边)} \\
&-e^{-at}u(-t) \longleftrightarrow \frac{1}{s+a}, \quad \text{Re}(s)<-a \quad \text{(左边)} \\
&te^{-at}u(t) \longleftrightarrow \frac{1}{(s+a)^2}, \quad \text{Re}(s)>-a \\
&-te^{-at}u(-t) \longleftrightarrow \frac{1}{(s+a)^2}, \quad \text{Re}(s)<-a \\
&e^{-a|t|} \longleftrightarrow \frac{2a}{s^2-a^2}, \quad -a<\text{Re}(s)<a \\
&t^n u(t) \longleftrightarrow \frac{n!}{s^{n+1}}, \quad \text{Re}(s)>0 \\
&\cos(\omega_0 t)u(t) \longleftrightarrow \frac{s}{s^2+\omega_0^2}, \quad \text{Re}(s)>0 \\
&\sin(\omega_0 t)u(t) \longleftrightarrow \frac{\omega_0}{s^2+\omega_0^2}, \quad \text{Re}(s)>0 \\
&e^{-at}\cos(\omega_0 t)u(t) \longleftrightarrow \frac{s+a}{(s+a)^2+\omega_0^2}, \quad \text{Re}(s)>-a \\
&e^{-at}\sin(\omega_0 t)u(t) \longleftrightarrow \frac{\omega_0}{(s+a)^2+\omega_0^2}, \quad \text{Re}(s)>-a
\end{align*}

\noindent\textbf{六、z变换}
\scriptsize
\textbf{定义:}$X(z) = \mathcal{Z}[x[n]] = \sum_{n=-\infty}^{\infty}x[n]z^{-n}$
\begin{itemize}
\itemsep=0pt
\item 线性:$\mathcal{Z}[ax[n]+by[n]] = aX(z)+bY(z)$,ROC至少为$R_1\cap R_2$
\item 时移:$\mathcal{Z}[x[n-n_0]] = z^{-n_0}X(z)$,ROC:$R$(可能除去$z=0$或$z=\infty$)
\item 尺度:$\mathcal{Z}[a^n x[n]] = X(\frac{z}{a})$,ROC:$|z/a|\in R$即$|z|\in|a|R$
\item z域微分:$\mathcal{Z}[nx[n]] = -z\frac{dX(z)}{dz}$,ROC:$R$
\item 时域卷积:$\mathcal{Z}[x[n]*y[n]] = X(z)Y(z)$,ROC至少为$R_1\cap R_2$
\item 差分:$\mathcal{Z}[x[n]-x[n-1]] = (1-z^{-1})X(z)$,ROC至少为$R\cap\{z\neq0\}$
\item 累加:$\mathcal{Z}[\sum_{k=-\infty}^{n}x[k]] = \frac{X(z)}{1-z^{-1}}$,ROC至少为$R\cap\{|z|>1\}$
\item 初值定理:$x[0] = \lim_{z\to\infty}X(z)$ (因果序列)
\item 终值定理:$\lim_{n\to\infty}x[n] = \lim_{z\to1}(z-1)X(z)$ (极点在单位圆内或$z=1$)
\end{itemize}
\begin{align*}
&\delta[n] \longleftrightarrow 1, \quad \text{全}z\text{平面} \\
&u[n] \longleftrightarrow \frac{z}{z-1}, \, |z|>1 \\
&-u[-n-1] \longleftrightarrow \frac{z}{z-1}, \, |z|<1 \\
&a^n u[n] \longleftrightarrow \frac{z}{z-a}, \, |z|>|a| \quad \text{(右边)} \\
&-a^n u[-n-1] \longleftrightarrow \frac{z}{z-a}, \, |z|<|a| \quad \text{(左边)} \\
&na^n u[n] \longleftrightarrow \frac{az}{(z-a)^2}, \, |z|>|a| \\
&-na^n u[-n-1] \longleftrightarrow \frac{az}{(z-a)^2}, \, |z|<|a| \\
&\cos(\omega_0 n)u[n] \longleftrightarrow \frac{z(z-\cos\omega_0)}{z^2-2z\cos\omega_0+1} \\
&\sin(\omega_0 n)u[n] \longleftrightarrow \frac{z\sin\omega_0}{z^2-2z\cos\omega_0+1}
\end{align*}

\end{multicols}

\newpage

\begin{multicols}{3}

\textbf{单边z变换:}
\scriptsize

\textbf{定义:}$X_u(z) = \mathcal{Z}_u[x[n]] = \sum_{n=0}^{\infty}x[n]z^{-n}$

\textbf{时移性质(关键区别):}
右移$n_0>0$:
$$\mathcal{Z}_u[x[n-n_0]] = z^{-n_0}X_u(z) + \sum_{k=0}^{n_0-1}x[k-n_0]z^{-k}$$
左移$n_0>0$:
$$\mathcal{Z}_u[x[n+n_0]] = z^{n_0}X_u(z) - z^{n_0}\sum_{k=0}^{n_0-1}x[k]z^{-k}$$
\textbf{差分方程求解:}单边z变换可直接包含初始条件

\textbf{应用场合:}单边适用于因果系统和初值问题;双边适用于一般信号分析


\textbf{单边z变换求解差分方程步骤:}
\begin{enumerate}
\itemsep=0pt
\item 对差分方程两边取单边z变换
\item 代入初始条件$x[0], x[-1], \ldots$
\item 求解$Y(z)$
\item 反变换得$y[n]$(通常为因果序列)
\end{enumerate}

\textbf{七、连续时间LTI系统分析}


\textbf{系统函数:}$H(s) = \frac{Y(s)}{X(s)} = \mathcal{L}[h(t)]$

\textbf{稳定性判据:}
\begin{itemize}
\itemsep=0pt
\item \textbf{BIBO稳定}:所有极点在左半平面,即$\text{Re}(p_j)<0$
\item \textbf{临界稳定}:极点在虚轴上
\item \textbf{不稳定}:至少一个极点在右半平面
\end{itemize}

\textbf{频率响应:}$H(j\omega) = H(s)|_{s=j\omega}$
\begin{itemize}
\itemsep=0pt
\item 幅频特性:$|H(j\omega)|$
\item 相频特性:$\angle H(j\omega)$
\end{itemize}

\textbf{时频特性:}

\textbf{无失真传输条件:}$y(t) = Kx(t-t_d)$
\begin{itemize}
\itemsep=0pt
\item 幅频:$|H(j\omega)| = K$(常数)
\item 相频:$\angle H(j\omega) = -\omega t_d$(线性相位)
\item 理想:$H(j\omega) = Ke^{-j\omega t_d}$
\end{itemize}

\textbf{群时延:}$\tau_g(\omega) = -\frac{d\theta(\omega)}{d\omega}$

其中$\theta(\omega) = \angle H(j\omega)$

物理意义:窄带信号包络的时延

\textbf{相时延:}$\tau_p(\omega) = -\frac{\theta(\omega)}{\omega}$

物理意义:载波的时延

\textbf{关系:}
\begin{itemize}
\itemsep=0pt
\item 线性相位:$\theta(\omega)=-\omega t_d$,则$\tau_g=\tau_p=t_d$
\item 非线性相位:$\tau_g \neq \tau_p$,产生失真
\end{itemize}

\textbf{部分分式展开:}

对于$H(s) = \frac{N(s)}{D(s)}$,若极点$p_i$为单极点:
$$H(s) = \sum_{i}\frac{r_i}{s-p_i} \quad \text{其中} \quad r_i = [(s-p_i)H(s)]_{s=p_i}$$

\textbf{零输入响应与零状态响应:}

\textbf{定义:}
\begin{itemize}
\itemsep=0pt
\item \textbf{零输入响应}$y_{zi}(t)$:输入为零,仅由初始条件产生的响应
\item \textbf{零状态响应}$y_{zs}(t)$:初始条件为零,仅由输入产生的响应
\item \textbf{全响应:}$y(t) = y_{zi}(t) + y_{zs}(t)$
\end{itemize}

\textbf{求解方法(单边拉普拉斯变换法):}

\textbf{1. 零状态响应:}
\begin{itemize}
\itemsep=0pt
\item 初始条件全为零
\item 对微分方程取单边拉氏变换
\item $Y_{zs}(s) = H(s)X(s)$,其中$H(s)$为系统函数
\item 反变换得$y_{zs}(t) = h(t)*x(t)$(卷积)
\end{itemize}

\textbf{2. 零输入响应:}
\begin{itemize}
\itemsep=0pt
\item 输入$x(t)=0$,仅考虑初始条件
\item 对微分方程取单边拉氏变换
\end{itemize}

\textbf{3. 全响应求解步骤:}
\begin{enumerate}
\itemsep=0pt
\item 对微分方程两边取单边拉氏变换(含初始条件)
\item 求解$Y(s) = Y_{zs}(s) + Y_{zi}(s)$
\item 部分分式展开
\item 反变换得$y(t)$
\end{enumerate}

\textbf{注意事项:}
\begin{itemize}
\itemsep=0pt
\item 初值$y(0^-), y'(0^-), \ldots$需从$t<0$状态确定
\item 若有冲激输入,需考虑$0^-$到$0^+$的跳变
\item 零状态响应由系统函数唯一确定
\item 零输入响应仅与初始条件和系统特征根有关
\end{itemize}

\noindent\textbf{八、离散时间LTI系统分析}
\scriptsize

\textbf{系统函数:}$H(z) = \frac{Y(z)}{X(z)} = \mathcal{Z}[h[n]]$

\textbf{稳定性判据:}
\begin{itemize}
\itemsep=0pt
\item \textbf{BIBO稳定}:所有极点在单位圆内,即$|p_j|<1$
\item \textbf{临界稳定}:极点在单位圆上
\item \textbf{不稳定}:至少一个极点在单位圆外
\end{itemize}

\textbf{频率响应:}$H(e^{j\omega}) = H(z)|_{z=e^{j\omega}}$

\textbf{差分方程与系统函数关系:}

差分方程:$\sum_{k=0}^{N}a_k y[n-k] = \sum_{k=0}^{M}b_k x[n-k]$

系统函数:$H(z) = \frac{Y(z)}{X(z)} = \frac{\sum_{k=0}^{M}b_k z^{-k}}{\sum_{k=0}^{N}a_k z^{-k}}$

\noindent\textbf{九、因果性与稳定性}
\scriptsize

\textbf{连续系统(s域):}
\begin{itemize}
\itemsep=0pt
\item \textbf{因果性}:$H(s)$是真有理函数(分子次数$\leq$分母次数)
\item \textbf{稳定性}:收敛域包含虚轴,所有极点$\text{Re}(p)<0$
\item \textbf{因果稳定}:收敛域为$\text{Re}(s)>\sigma_0$且$\sigma_0<0$
\end{itemize}

\textbf{离散系统(z域):}
\begin{itemize}
\itemsep=0pt
\item \textbf{因果性}:$H(z)$收敛域为$|z|>r$(某个圆外)
\item \textbf{稳定性}:收敛域包含单位圆$|z|=1$,所有极点$|p|<1$
\item \textbf{因果稳定}:收敛域为$|z|>r$且$r<1$
\end{itemize}

\noindent\textbf{十、系统连接}
\scriptsize

\textbf{串联:}$H(s) = H_1(s)H_2(s)$ \quad 或 \quad $H(z) = H_1(z)H_2(z)$

\textbf{并联:}$H(s) = H_1(s)+H_2(s)$ \quad 或 \quad $H(z) = H_1(z)+H_2(z)$

\textbf{反馈:}$H(s) = \frac{H_1(s)}{1\pm H_1(s)H_2(s)}$ \quad ($+$负反馈,$-$正反馈)


\noindent\textbf{十一、采样理论}
\scriptsize

\textbf{冲激串采样:}
\begin{itemize}
\itemsep=0pt
\item 采样信号:$x_p(t) = x(t)p(t) = x(t)\sum_{n=-\infty}^{\infty}\delta(t-nT)$
\item 频域:$X_p(j\omega) = \frac{1}{T}\sum_{k=-\infty}^{\infty}X(j(\omega-k\omega_s))$,$\omega_s=\frac{2\pi}{T}$
\item 频谱周期延拓,周期为$\omega_s$
\end{itemize}

\textbf{采样定理(Nyquist):}
\begin{itemize}
\itemsep=0pt
\item 若$x(t)$带限于$\omega_M$,即$X(j\omega)=0$,$|\omega|>\omega_M$
\item 采样频率$\omega_s \geq 2\omega_M$(Nyquist率),可无失真恢复
\item 恢复:理想低通滤波器$H_r(j\omega) = \begin{cases} T, & |\omega|<\omega_c \\ 0, & |\omega|\geq\omega_c \end{cases}$
\item 其中$\omega_M < \omega_c < \omega_s-\omega_M$
\end{itemize}

\textbf{零阶保持(ZOH)采样:}
\begin{itemize}
\itemsep=0pt
\item $x_0(t) = \sum_{n=-\infty}^{\infty}x(nT)[\text{rect}(\frac{t-nT}{T}-\frac{1}{2})]$
\item 频域:$X_0(j\omega) = H_0(j\omega)X_p(j\omega)$
\item $H_0(j\omega) = T\text{sa}(\frac{\omega T}{2})e^{-j\omega T/2}$
\end{itemize}

\textbf{线性内插:}
\begin{itemize}
\itemsep=0pt
\item $x_1(t) = \sum_{n=-\infty}^{\infty}x(nT)\text{tri}(\frac{t-nT}{T})$
\item 频域:$X_1(j\omega) = H_1(j\omega)X_p(j\omega)$
\item $H_1(j\omega) = T[\frac{\sin(\omega T/2)}{\omega T/2}]^2 = T\text{sa}^2(\frac{\omega T}{2})$
\end{itemize}

\textbf{离散时间处理连续信号:}
\textbf{C/D转换(连续到离散):}
   \begin{itemize}
   \itemsep=0pt
   \item 采样:$x[n] = x_c(nT)$
   \item 频域关系:$X(e^{j\omega}) = \frac{1}{T}\sum_{k=-\infty}^{\infty}X_c(j\frac{\omega-2\pi k}{T})$
   \end{itemize}
\textbf{离散时间处理:}
   \begin{itemize}
   \itemsep=0pt
   \item 离散系统:$Y(e^{j\omega}) = H(e^{j\omega})X(e^{j\omega})$
   \item 等效连续频率:$\Omega = \omega T$
   \end{itemize}
\textbf{D/C转换(离散到连续):}
   \begin{itemize}
   \itemsep=0pt
   \item 理想重建:$y_c(t) = \sum_{n=-\infty}^{\infty}y[n]\frac{\sin(\pi(t-nT)/T)}{\pi(t-nT)/T}$
   \item 零阶保持:$y_c(t) = \sum_{n=-\infty}^{\infty}y[n]\text{rect}(\frac{t-nT-T/2}{T})$
   \end{itemize}
\textbf{整体系统:}
   \begin{itemize}
   \itemsep=0pt
   \item 等效连续系统:$H_{eff}(j\Omega) = H(e^{j\Omega T})$,$|\Omega|<\pi/T$
   \item 采样频率足够高时近似连续滤波器
   \end{itemize}


\textbf{混叠(Aliasing):}
\begin{itemize}
\itemsep=0pt
\item 当$\omega_s < 2\omega_M$时,频谱混叠
\item 高频成分"伪装"成低频,无法恢复
\item 解决:预滤波(抗混叠滤波器)
\end{itemize}

\textbf{离散时间采样(多速率信号处理):}

\textbf{1. 抽取(Decimation/下采样):}
\begin{itemize}
\itemsep=0pt
\item 定义:$y[n] = x[Mn]$,每$M$个样本取一个($M$为整数)
\item 频域:$Y(e^{j\omega}) = \frac{1}{M}\sum_{k=0}^{M-1}X(e^{j(\omega-2\pi k)/M})$
\item 频谱周期延拓并压缩,采样率降低$M$倍
\item 抗混叠:先低通滤波$H(e^{j\omega})$,截止频率$\omega_c = \pi/M$,再抽取
\end{itemize}

\textbf{2. 零值插入(Zero-padding/上采样):}
\begin{itemize}
\itemsep=0pt
\item 定义:$y[n] = \begin{cases} x[n/L], & n=0,\pm L,\pm 2L,\ldots \\ 0, & \text{其他} \end{cases}$($L$为整数)
\item 频域:$Y(e^{j\omega}) = X(e^{j\omega L})$
\item 频谱扩展$L$倍,产生$L-1$个镜像
\item 需后接低通滤波器去除镜像,截止频率$\omega_c = \pi/L$
\end{itemize}

\textbf{3. 内插(Interpolation/插值):}
\begin{itemize}
\itemsep=0pt
\item 定义:零值插入后低通滤波,$y[n] = (x_L[n]*h[n])$
\item 低通滤波器:$H(e^{j\omega}) = \begin{cases} L, & |\omega|\leq\pi/L \\ 0, & \pi/L<|\omega|\leq\pi \end{cases}$
\item 理想内插:$h[n] = \frac{\sin(\pi n/L)}{\pi n/L}$
\item 采样率提高$L$倍,插入$L-1$个样本
\end{itemize}

\textbf{4. 离散时间冲激串采样:}
\begin{itemize}
\itemsep=0pt
\item 周期序列:$p[n] = \sum_{k=-\infty}^{\infty}\delta[n-kN]$
\item 采样:$y[n] = x[n]p[n]$
\item 频域:$Y(e^{j\omega}) = \frac{1}{N}\sum_{k=0}^{N-1}X(e^{j(\omega-2\pi k/N)})$
\item 频谱周期延拓,周期$2\pi/N$
\end{itemize}

\textbf{注意:}
\begin{itemize}
\itemsep=0pt
\item 审题
\item 求单位冲激响应还是单位阶跃响应
\item 注意题目中是否隐含因果性和稳定性
\item 注意拉氏变换和z变换的收敛域(每一步都要求)
\item 求响应前注意收敛域是否存在,若不存在用本征函数或卷积

\end{itemize}

\end{multicols}

\end{document}